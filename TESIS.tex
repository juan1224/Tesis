\documentclass[12pt,letterpaper]{article}
\usepackage[T1]{fontenc}
\usepackage[utf8]{inputenc}
\usepackage[spanish]{babel}
\usepackage{amsmath}
\usepackage{amsfonts}
\usepackage{amssymb}
\usepackage{graphicx}
\usepackage{xcolor}
\usepackage{times}
\usepackage{float}
 
\usepackage{color}
\definecolor{gray97}{gray}{.97}
\definecolor{gray75}{gray}{.75}
\definecolor{gray45}{gray}{.45}
 
\usepackage{listings}
\lstset{ frame=Ltb,
     framerule=0pt,
     aboveskip=0.5cm,
     framextopmargin=3pt,
     framexbottommargin=3pt,
     framexleftmargin=0.4cm,
     framesep=0pt,
     rulesep=.4pt,
     backgroundcolor=\color{gray97},
     rulesepcolor=\color{black},
     %
     stringstyle=\ttfamily,
     showstringspaces = false,
     basicstyle=\small\ttfamily,
     commentstyle=\color{gray45},
     keywordstyle=\bfseries,
     %
     numbers=left,
     numbersep=15pt,
     numberstyle=\tiny,
     numberfirstline = false,
     breaklines=true,
   }
 % minimizar fragmentado de listados
\lstnewenvironment{listing}[1][]
   {\lstset{#1}\pagebreak[0]}{\pagebreak[0]}
 
\lstdefinestyle{consola}
   {basicstyle=\scriptsize\bf\ttfamily,
    backgroundcolor=\color{gray75},
   }
 
\lstdefinestyle{C}
   {language=C,
   }  
   
\usepackage{listings}
\lstset{basicstyle=\ttfamily,
  showstringspaces=false,
  commentstyle=\color{red},
  keywordstyle=\color{blue},
}
\usepackage{pdfpages}
\usepackage[left=2cm,right=2cm,top=2cm,bottom=2cm]{geometry}
\author{Juan Jose Martinez Ulloa}

\begin{document}
\includepdf[pages=1]{PORTADA}
\section*{Resumen.}
\section*{Abstract.}
\section*{Agradecimientos.}
\tableofcontents % indice de contenido
\section{Introducción}
\subsection{Planteamiento del Problema}
El cáncer de mama es una enfermedad compleja y heterogénea con más de 1,300,000 casos y 450,000 muertes cada año en todo el mundo. Esta enfermedad se caracteriza por diferentes ápectos biológicos como desregulazación de la expresión génica, alteraciones genómicas del ADN, etc. Todo esto da lugar al inicio y desarrollo del carcinoma de mama. En éstos últimos años el uso de datos ómicos, como los basados en  microarreglos (microarrays) y secuenciación, esta en su pleno () en el campo de la biomedicina. Todos estos datos permiten estudiar enfermedades desde un punto de vista biomolecular. Con esto, se ofrecen grandes oportunidades para mejorar tanto la comprención de la enfermedad, como el desarrollo de nuevos métodos para el diagnóstico y tratamiento del paciente, sin embargo, el análisis de estos datos producidos por estas tecnologías, es bastante complejo por lo que es necesario la aplicación de avanzadas técnicas de análisis y cálculos computacionales que permiten obtener la información biológica disponibles. Hasta el día de hoy todos estos datos óptenidos de diversos experimentos, datos de muy alta calidad, datos clinicamente bien anotados, y una gran cantidad de datos de canceres analizados con el fin de encontrar anomalias recurrentes que sean importantes de la enfermedad y estos datos se guardan en diversas plataformas que permiten tener uns gran cantidad de información , pero estos datos no son tan faciles de analizar, por ello, la bioinformática ayuda a manejar estructurar y organizarla para que sea mas fácil de comprender. Circos plot es una de las herramientas que existen para la visualización datos, ideal para explorar las relaciones entre objetos y posiciones. Esta herramienta es flexible, aunque originalmente fue diseñado para visualizar datos genómicos , se puede crear figúras a partir de datos en cualquier campo, desde la genómica hasta la visualización de la migración al arte matemático. Esta herramienta puede ser automatizada. Está controlado por archivos de configuración de texto plano, lo que lo hace díficil su incorporación en \textit{pipeline} de adquisición de datos, análisis e informes. Todo esto hace dificil su utilización en instalacion y desarrollo de la visualización. 
\subsection{Justificación}
La cantidad de información generada a través de las tecnologías de secuenciación masiva son enormes. La integración de toda esa información puede ayudar a tomar decisiones de índole biomédica e incluso clínica. 
Aunque existen plataformas ya conocidas capaces de presentar esa información de un modo amigable, la programación de dichas herramientas continúa siendo complicada para un usuario final. Es por esto, que generar una herramienta con gran capacidad de visualización, integración de información y facilidad de implementación para un usuario final es de la mayor importancia. 

Con el fin de mejorar el rendimiento y la cantidad de tiempo de los investigadores del Instituto Nacional de Medicina Genómica, es fundamental sistematizar este software para reducir el tiempo de programación y la investigación.
La sistematización 	de circos plot, brindará la posibilidad de que el investigador ahorre en tiempo de programación o en dado caso que no se conozca nada del uso nativo de circos plot, leer todo el manual de uso de dicho software, para que ocupe su mayor de tiempo en la investigación y solo tome varios minutos para diseñar su grafica circular llamada circos plot.


%Agregarlos a otros flujos de trabajo
%A la fecha se utilizaban 2 platoformas pero con el desarrollo se implementaran 4 
\subsection{Delimitación}
El desarrollo de tecnologías de secuenciación masiva es abrumador. Para poder integrar toda la información proveniente de dichas tecnologías es necesario contar con herramientas adecuadas para la obtención, análisis y visualización de las tecnologías anteriormente mencionadas.

Ante esta problemática, resulta altamente relevante generar una herramienta capaz de condensar en un solo golpe de vista varias capas de información, para que, de este modo, se pueda analizar con un mayor detalle los datos obtenidos.

Circos plot es una herramienta de apoyo visual que permite lograr la visualización a nivel de genoma completo varios tipos diferentes de datos: Expresión, mutaciones, metilación, citobandas, etc., de una manera amigable para el usuario final. 

Aunque circos plot tiene grandes ventajas en cuanto a la visualización, su programación es altamente complicada, por lo que generar un back-end que integre de modo sencillo varios tipos de datos, será de la mayor utilidad. 

\subsection{Hipótesis}
El desarrollo de una herramienta de Visualización integral permitirá el análisis de la información de los datos derivados de múltiples plataformas.
\subsection{Objetivos}
\subsubsection{Objetivo General}
Implemetar una herramienta de visualización que integre datos genómicos derivados de múltiples plataformas.
\subsubsection{Objetivos Específicos}
\begin{itemize}
\item Realizar el pretatamiento de los datos de trascriptoma (Microarreglos).
\item Realizar el pretatamiento de los datos de Metilación.
\item Desarrollar algoritmo computacional para hacer la búsqueda en la referencia del genoma humano con los datos de trascriptoma y metilación.
\item Desarrollar software para preprocesar datos de distintos tipos de tecnologías genómicas.
\item Desarrollar software para integrar al backbone de circos plot varias capas de información proveniente del paso anterior.
\item Generar opciones de visualización de fácil manejo para el usuario final.

\end{itemize}
\subsection{Aportaciones de la tesis}
La presente Tesis aportara
\section{Estado del Arte}
\section{Tecnologías genómica.}
Las tecnologías genómicas es el conjunto de herramientas orientadas al estudio integral del funcionamiento, contenido,  evolución del genoma. Es una de las áreas más vanguardistas de la biología. La genómica usa conocimientos derivados de distintas ciencias como la biología molecular, la bioquímica, la informática, la estadística, las matemáticas y la física.
Para entender un poco más de estas tecnologias y de los datos que se obtiene de las antes mencionadas, hablaremos de las tecnologias genómicas que son: Microarreglos y Metilación.
\section{Microarreglos}
Un microarreglo de ADN (ácido desoxirribonucleico) tambien llamado DNA chip, tambien llamado \textbf{oligonucleotido de DNA chip o gene chip} que consiste en pequeños fragmentos de ADN de los cuales representa un gen diferente \cite{VallinPlous2007}, es una superficie sólida a la cual se une una colección de fragmentos de ADN. Las superficies empleadas para fijar el ADN son muy variables y pueden ser de vidrio, plástico e incluso de silicona. Los chips de ADN se usan para analizar la expresión diferencial de genes.  Su funcionamiento consiste, básicamente, en medir el nivel de hibridación entre la sonda específica (probe, en inglés), y la molécula diana (target), y se indican generalmente mediante fluorescencia y a través de un análisis de imagen, lo cual indica el nivel de expresión del gen.	\\
Por lo tanto, los microarreglos son una potente fuente de obtención de perfiles de expresión de genes sometidos a diferente condiciones. Identificar los patrones de los niveles de expresión será muy útil para compararlos y poder estudiar las respuestas de los genes. \\
Aplicando una serie de procesos experimentales y computacionales sobre los microarreglos se obtiene una una matriz numerica bidimensional que consta de los genes de poblaciones distintas como individuos y de las condiciones experimentales a las que expusieron as células como variables en el caso que se quiera estudiar a los genes, o a la inversa, si es que se quiere realizar un estudio comparativo de las condiciones a que se somete. Cada uno de los valores de la matriz representa el nivel de expresión de un determinado gen bajo una cierta condición experimental. \\
Estas matrices son de grandes dimensiones puesto que existen una gran cantidad de condiciones experimentales y genes. En la figura 1 se puede observar el modelo de la presentación de los microarreglos. Cada fila representa un gen, el cual debe de ser indicado, cada columna represnta una condición experimental, cuyo nombre tambien debe de ser identicado. Los valores de la matriz son los niveles de expresión de los genes para la condicion experimental.
\begin{figure}[H]
\begin{center}
\includegraphics[width=0.7\textwidth]{matrix.png}
\end{center}
\caption{Chip de un Microarreglo}
\end{figure}
Dado que realizar un análisis de estas matrices de grandes dimensiones es una tarea prácticamente imposible, se hace necesarias técnicas computacionales que permitan analizar todos estos datos y entonces realizar el análisis biológico.\\
Actualmente existen diferentes bases de datos a nuestro alcance a través de Internet que unifican y facilitan toda esta información genética además de ofrecer diversas herramientas para el análisis de esta gran cantidad de información. Algunas de estas bases de datos por ejemplo son las que hay en el EMBL (European Molecular Biology Laboratory), el SIB (Swiss Institute of Bioinformatics), el EBI (European Bioinformatics Institute) o el NCBI (National Center for Biotechnology Information). El EBI y el NCBI son los que más información contienen y por lo tanto los más utilizados.
El tamño de los microarreglos es de 1.28 cm x 1.28 cm, hay 500,000 ubicaciones en cada matriz y por lo general tiene millones de cadenas de ADN construidas en cada ubicación, cada cadena contiene 25 pares bases (Figura 1).
\begin{figure}[H]
\begin{center}
\includegraphics[width=0.7\textwidth]{Genechip.png}
\end{center}
\caption{Chip de un Microarreglo.}
\end{figure}
En estos chips se imprimen las secuencias biologicas en un chip, de manera que se puede cuantificar el transcripción en una matriz numérica.	
\subsection{Formato}
Los archivos están disponibles en un formato de valores separados por comas (CSV). Estos son archivos de texto sin formato con cada fila terminada por un carácter de nueva línea. Los datos en campos separados están entre comillas y separados por comas. Ninguno de los campos de datos contiene ninguno de estos caracteres: comillas, nueva línea, retorno de carro o tabulación.\\

Estos archivos se usen principalmente en aplicaciones de hojas de cálculo y programas de bases de datos (como bases de datos SQL). Los datos estan formateados de tal manera que estos dos usos sean relativamente fáciles. Se tiene en cuenta que algunos de los archivos y los campos de datos en ellos son grandes. 

La primera fila de cada archivo contiene los títulos de los campos que figuran en las filas siguientes.\\

Cada fila después de la primera fila contiene anotaciones para un solo conjunto de sondas. Todas las anotaciones para ese conjunto de sonda están contenidas en esa única fila. En algunos campos, como las anotaciones de dominio de proteínas, puede haber más de una anotación para un único conjunto de sondas. En este caso, los valores múltiples están separados por la cadena '///'.\\

En muchos tipos de anotaciones, los subcampos están separados por '//'. Por ejemplo,  una anotación para un "GO Biological Process" puede aparecer como "7155 // cell adhesion // predicted / computed".  En este caso, las secciones corresponden a "ID // Descripción // Evidencia", pero el significado de los subcampos varía entre los diferentes tipos de anotación, como se describe a continuación.\\

Los campos vacíos se indican con '- - -' . El hecho de utilizar una cadena de este tipo en lugar de dejar el campo vacío es que hace que la naturaleza columnar de los datos sea más visible en ciertos programas de hoja de cálculo.
Algunas columnas en algunos archivos no contienen datos. Para ayudar a los usuarios a combinar datos de varios archivos, dichas columnas vacías no se eliminan. Por lo tanto, cada archivo tiene las mismas columnas en el mismo orden.\\

Algunos campos, como "Chip", contienen el mismo valor para cada conjunto de sonda en un archivo. Aunque estos datos son redundantes en cualquier archivo individual, son útiles para los usuarios que combinan datos de varios archivos.
\subsection{Plataformas}
Actualmente existen diferentes bases de datos a nuestro alcance a través de Internet
que unifican y facilitan toda esta información genética además de ofrecer diversas
herramientas para el análisis de esta gran cantidad de información. Algunas de estas bases
de datos por ejemplo son las que hay en el EMBL (European Molecular Biology Laboratory),
el SIB (Swiss Institute of Bioinformatics), el EBI (European Bioinformatics Institute) o el NCBI
(National Center for Biotechnology Information). El EBI y el NCBI son los que más
información contienen y por lo tanto los más utilizados. 

Algunas de la plataformas donde se aloja los datos de microarreglos son varia y una de las más conocidas en la comunidad cientifia es TCGA(The Cancer Genome Atlas )
TCGA es una colaboración entre el Instituto Nacional del Cáncer (NCI) y el Instituto Nacional de Investigación del Genoma Humano (NHGRI) que ha generado mapas completos y multidimensionales de los principales cambios genómicos en 33 tipos de cáncer. El conjunto de datos TCGA, que comprende más de dos petabytes de datos genómicos, se ha hecho públicamente disponible, y esta información genómica ayuda a la comunidad de investigación del cáncer a mejorar la prevención, el diagnóstico y el tratamiento del cáncer.

Alguna de las plataformas son:

\begin{itemize}
\item Illumina
\item Affymetrix SNP 6.0
\end{itemize}


\subsection{Metilación}
La metilación es un proceso epigenetico que participa en la regulacion de la expresión genica de dos maneras, direcatmente al impedir de la unión de factores de transcripción, e indirectamente proporcionando la estructura "cerrada" de la cromatina\cite{Mesa-Cornejo, Viviana Matilde, Barros-Núñez, Patricio, & Medina-Lozano, Claudina2006}.

\section{Propuesta de Solución}
%plticado.
La solución ante la problemática planteada, consiste fundamentalmente en incoporporar a circos plot a un interfaz de usuario amigable para el usuario final agregando la mayoria de las funciones del software para su buen funcionamiento, implementando de manera subyacente un programa escrito en python para el mapeo de los datos con el diccionario de referncia del genoma humano para obtener el formato que se requiere para correr el software.
Así mismo realizar incorporar circos plot a Galaxy que una platorma que contiene herramientas para análisis de datos para investigación cientifica y de esta manera implementar circos plot en el laboratorio de genómica computacional en el Instituto Nacional de Medicina Genómica

\section{Metodología}
\subsection{Diseño de estudio}

\begin{figure}[H]
\begin{center}
\includegraphics[width=0.9\textwidth]{METODOLOGIA.png}
\end{center}
\caption{Metodología}
\end{figure}


\subsection{Circos plot}
\subsubsection{¿Qué es Circos plot?}
Circos fue diseñado inicialmente para mostrar datos genómicos (particularmente genómica del cáncer y genómica comparativa) y biología molecular. Tiene características específicas que abordan los retos típicos en el dibujo de este tipo de datos, que tienden a ser muy escasa y abarcan un gran número de escalas de longitud.\\

Circos fue concebido originalmente para visualizar datos gen\'{o}micos tales como alineaciones y variaci\'{o}n estructural. Con el tiempo, se agreg\'{o} soporte para pistas de datos 2D, tales como trazados de l\'{i}nea, dispersi\'{o}n, mapa de calor e histograma.\\

A medida que crecía la popularidad de Circos, provocada por una infografía de página completa del New York Times , comenzó a utilizarse para visualizar otros datos, no sólo genómicos.\\

Es una herramienta de visualización para facilitar la identificación y análisis de similitudes y diferencias que surgen de las comparaciones de genomas. La herramienta es eficaz en la visualización de la variación en la estructura del genoma y, en general, cualquier otro tipo de relaciones de posición entre los intervalos genómicos. Tales datos se producen rutinariamente mediante alineaciones de secuencias, matrices de hibridación, mapeo del genoma y estudios de genotipado. Circos utiliza un diseño de ideograma circular para facilitar la visualización de relaciones entre pares de posiciones mediante el uso de cintas, que codifican la posición, el tamaño y la orientación de los elementos genómicos relacionados. Circos es capaz de mostrar datos como diagramas de dispersión, líneas y histogramas, mapas de calor, mosaicos, conectores y texto. Las imágenes de mapa de bits o vectoriales pueden crearse a partir de entradas de datos estilo GFF y archivos de configuración jerárquica, que pueden generarse fácilmente mediante herramientas automatizadas, lo que hace que Circos sea adecuado para un despliegue rápido en el análisis de datos y en las tuberías de generación de informes.\\

Circos ha aparecido en muchas publicaciones, tanto científicas como generales. Ha cambiado la forma en que la comunidad científica visualiza alteraciones genómicas (cambios en un genoma en el tiempo, o diferencias entre dos o más genomas). Una aplicación oportuna de este enfoque es la creación de cifras eficaces que muestran cómo los genomas del cáncer difieren de los sanos (por ejemplo, COSMIC: Censo de las mutaciones somáticas en el cáncer). \\

La comunidad científica biológica ha adoptado a Circos de todo corazón. Por ahora, Circos ha aparecido en las portadas de las publicaciones de Nature y Science , que son las principales revistas científicas del mundo como por ejemplo Bioinformatics, GenomeBiology, nature, Science, Nucleic Acids Research, American Scientis, Genome Reserch, PortFolio, Wired, PNAS, PLOS, entre otras.\\

Circos plot es una idea original de \textbf{Martin Krzywinski} y de un grupo de trabajo atras de el. El articulo donde habla sobre esta nueva herramienta de visualizaci\'{o}n \emph{\textbf{(Circos: an information aesthetic for comparative genomics)}} fue publicado el 9 de septiembre del 2009 y actualmente tiene 2973 citas, esto quiere decir que es una herramienta verdaderamente utilizada no solo en la biolog\'{i}a y todas sus ramas si no en muchas otros campos de la ciencia.\\

\textbf{Instalación de circos}
Primero, descargamos Circos \texttt{http://circos.ca/software/download}. El contenido de la distribución se describe a continuación.
No necesitamos mover o editar ningún archivo en la distribución principal.
\begin{lstlisting}[language=bash,caption={Suponiendo que desea instalar en ROOT=~/software/circos},style=consola]
$ cd ~
$ mkdir software
$ mkdir software/circos
$ cd software/circos
# Descargar Circos y colocalo en el directorio software/circos
$ wget http://circos.ca/distribution/circos-0.69-6.tgz
# Descargar la versión mas actual (Recomendación).
#Descomprime 
$ tar xvfz circos-0.69-6.tgz
...
circos-0.69-6/data/karyotype/karyotype.arabidopsis.txt
circos-0.69-6/data/karyotype/karyotype.zeamays.txt
circos-0.69-6/data/karyotype/karyotype.oryzasativa.txt
# Crea un enlace simbolico a current 
$ ln -s circos-0.69-6 current
# Comprobamos si se creo el enlace simbolico.
$ ls -lh
drwxrwxr-x 9 juanjo juanjo 4.0K nov 29 10:36 circos-0.69-6/
-rw-rw-r-- 1 juanjo juanjo  22M nov 29 10:36 circos-0.69-6.tgz
lrwxrwxrwx 1 juanjo juanjo   13 nov 29 10:30 current -> circos-0.69-6/
# Borramos el archivo tgz, si ustede quiere
\end{lstlisting}
Para instalar los módulos GD y Perl en Ubuntu, usamos apt-get.\\
\begin{lstlisting}[language=bash, style=consola]
$ sudo apt-get -y install libgd2-xpm-dev
\end{lstlisting}
\textbf{CORRIENDO CIRCOS}\\ 
Circos utiliza banderas de línea de comandos, que son obligatorias. Por lo menos, debe especificar el archivo de configuración de imagen usando -conf.\\
Es una buena idea agregar el \textbf{bin/} directorio en la distribución para PATH que pueda ejecutar \textbf{bin/circos} desde cualquier lugar.\\
Añadimos al \textbf{root=~/software/circos/current} como se describió anteriormente, añadimos esto a nuestro \textbf{~/.bashrco} \textbf{~/.bash-profile}.\\ 
\begin{lstlisting}[language=bash, style=consola]
$ export PATH=~/software/circos/current/bin:$PATH
\end{lstlisting}
Ejecutamos explícitamente cualquiera \textbf{~/.bashrc} \textbf{~/.bash-profile} para que esto surta efecto\\
\begin{lstlisting}[language=bash, style=consola]
$ .~/ .bashrc
# o 
$ .~/ .bash_profile
\end{lstlisting}
Finalmente, probamos que nuestro PATH ha sido modificado,\\
\begin{lstlisting}[language=bash, style=consola]
$ cd ~
$ echo $PATH
~/software/circos/current/bin: ...
$ which circos
~/software/circos/current/bin/circos
\end{lstlisting}
\textbf{Revisando si faltan módulos Perl}\\
Verificamos si tenemos algún módulo faltante\\
\begin{lstlisting}[language=bash, style=consola]
$ circos -modules
ok       1.36 Carp
ok       0.38 Clone
ok       2.63 Config::General
ok       3.56 Cwd
ok      2.158 Data::Dumper
ok       2.54 Digest::MD5
ok       2.85 File::Basename
ok       3.56 File::Spec::Functions
ok     0.2304 File::Temp
ok       1.51 FindBin
ok       0.39 Font::TTF::Font
ok       2.53 GD
ok        0.2 GD::Polyline
ok       2.45 Getopt::Long
ok       1.16 IO::File
ok      0.413 List::MoreUtils
ok       1.41 List::Util
ok       0.01 Math::Bezier
ok     1.9997 Math::BigFloat
ok       0.07 Math::Round
ok       0.08 Math::VecStat
ok       1.03 Memoize
ok    1.53_01 POSIX
ok       1.26 Params::Validate
ok       1.64 Pod::Usage
ok       2.05 Readonly
ok 2016060801 Regexp::Common
ok       2.64 SVG
ok       1.19 Set::IntSpan
ok     1.6611 Statistics::Basic
ok    2.53_01 Storable
ok       1.20 Sys::Hostname
ok       2.03 Text::Balanced
ok       0.60 Text::Format
ok     1.9726 Time::HiRes
\end{lstlisting}
Cuando tenemos estas cosas ya tenemos circos plot instalado en nuestro SO
\subsection{Galaxy}
Galaxy es una plataforma abierta basada en la web para la investigación biomédica computacional accesible, reproducible y transparente.\\
\textbf{Accesible:} los usuarios sin experiencia en programación pueden especificar fácilmente parámetros y ejecutar herramientas y flujos de trabajo.\\
\textbf{Reproducible:} Galaxy captura información para que cualquier usuario pueda repetir y comprender un análisis computacional completo.\\
\textbf{Transparente:} los usuarios comparten y publican análisis a través de la web y crean páginas, documentos interactivos y basados ​​en la web que describen un análisis completo.\\
Para obtener instalado galaxy necesitamos seguir los pasos siguientes:\\
\textbf{Requisitos}
\begin{itemize}
\item UNIX / Linux o Mac OSX
\item Python 2.7
\end{itemize}
\textbf{Empezar}\\
Para producción o usuario único\\
\textbf{Clonar Galaxy desde GitHub}\\
\begin{lstlisting}[language=bash, style=consola]
$ git clone -b release_17.09 https://github.com/galaxyproject/galaxy.git
\end{lstlisting}
\textbf{Comenzarlo}
\\
Galaxy requiere algunas cosas para ejecutar: un virtualenv, archivos de configuración y módulos dependientes de Python. Sin embargo, iniciar el servidor por primera vez creará / adquirirá estas cosas según sea necesario. Para iniciar Galaxy, simplemente ejecute el siguiente comando en una ventana de terminal:
\begin{lstlisting}[language=bash, style=consola]
$ cd ~
$ cd /galaxy
# En contadas ocaciones se requiere dara permisos de lectura escritura y ejecución para ello ponmos en terminal:
$ chmod -R 777 run.sh
$ sh run.sh
\end{lstlisting}
\begin{figure}[h]
\begin{center}
\includegraphics[width=0.9\textwidth]{run.png}
\end{center}
\caption{Consola corriendo galaxy.}
\end{figure}
Esto iniciará el servidor Galaxy en el host local y el puerto 8080. Luego se puede acceder a Galaxy desde nuestro navegador web en \textbf{http:// localhost:8080.} Después de comenzar, el servidor de Galaxy imprimirá la salida a la ventana del terminal. Para detener el servidor Galaxy, se puede presionar las  Ctrl+C en la ventana de la terminal desde la que se está ejecutando Galaxy.
\begin{figure}[h]
\begin{center}
\includegraphics[width=0.9\textwidth]{galaxy.png}
\end{center}
\caption{Pantalla principal de Galaxy.}
\end{figure}
\textbf{Próximos pasos}
Convertirse en administrador\\
Para controlar Galaxy a través de la interfaz de usuario (instalación de herramientas, administración de usuarios, creación de grupos, etc.), los usuarios deben convertirse en administradores . Solo los usuarios registrados pueden convertirse en administradores. Para otorgar privilegios de administrador a un usuario, se completaron los siguientes pasos:\\
\begin{itemize}
\item En el directorio config/ viene el archivo de configuración pero lo encontramos como \textbf{galaxy.ini.sample} pero galaxy no lo reconoce así pero podemos tenerlo simplemente copiarlo como se muestra aqui:\\
\begin{lstlisting}[language=bash, style=consola]
$ cd /galaxy/config
$ cp galaxy.ini.sample galaxy.ini
$ vi galaxy.ini
\end{lstlisting}
\item Agregamos el correo electrónico de inicio de sesión Galaxy del usuario al archivo de configuración config/galaxy.ini. Como se muestra aquí:\\
\begin{lstlisting}[language=bash, style=consola]
# Esta linea viene comentada por lo que hay que descomentarlar y ponemos:
admin_users = jjmartinez@inmegen.edu.mx
\end{lstlisting}
\item Reinicié Galaxy después de modificar el archivo de configuración para que los cambios surtan efecto.
\end{itemize}



\begin{thebibliography}{XXX0000}
  \bibitem{VallinPlous2007} Microarreglos de ADN y sus aplicaciones en investigaciones biomédicas. Revista CENIC. Ciencias Biológicas, 38 (2), 132-135. 
  \bibitem {Mesa-Cornejo, Viviana Matilde, Barros-Núñez, Patricio, & Medina-Lozano, Claudina2006} Metilación del ADN: marcador diagnóstico y pronóstico de cáncer. Gaceta médica de México, 142(1), 81-82. 
\end{thebibliography} 	
\end{document}
